\documentclass[12pt]{article}
\usepackage{charter} % font
\usepackage[margin=1in]{geometry} % margin
\usepackage{varioref} % "on the next page" label decoration
\usepackage{hyperref} % hyperlinks
\usepackage{enumitem} % Enumeration
\usepackage{float} % table placement

\usepackage[lined,dotocloa]{algorithm2e} % Pseudocode
\labelformat{algocf}{\textit{alg.}\,(#1)}

% Questions
\newenvironment{QandA}{\begin{enumerate}[label=\bfseries\alph*.]\bfseries}
{\end{enumerate}}
\newenvironment{answered}{\par\normalfont}{}
\usepackage{lipsum}

%%%%%%%%%%%%%%%%%%%%%%%%%%%%%%%%%%%%%%%%%%%%%%%%%%%%%%%%%%%%%%%%%%%%%%%%%%%%%%%%

\title{%
    \textbf{Assignment 6 \\ 
        Huffman Coding \\
\large DESIGN DOCUMENT} }
\author{Zack Traczyk \\ CSE13S - Spring 2021}
\date{Due: May 23\textsuperscript{th} at 11:59 pm}

%%%%%%%%%%%%%%%%%%%%%%%%%%%%%%%%%%%%%%%%%%%%%%%%%%%%%%%%%%%%%%%%%%%%%%%%%%%%%%%%

\begin{document}

\maketitle

\section{Objective}


\section{Given}

\begin{itemize}
    \item{Header files for stack}
    \item{Pseudocode for recursive search}
\end{itemize}

%%%%%%%%%%%%%%%%%%%%%%%%%%%%%%%%%%%%%%%%%%%%%%%%%%%%%%%%%%%%%%%%%%%%%%%%%%%%%%%%

\section{Programs}

\subsection{Encode}

\begin{itemize}
    \item{-h : Command line options}
    \item{-i infile : The file containing bytes to be encoded (default is stdin)}
    \item{-o outfile : The output file to store encoded bytes (default is stdout)}
\end{itemize}

\subsection{Decode}

\begin{itemize}
    \item{-h : Command line options }
    \item{-v : Prints statists of the decoding process to stderr}
    \item{-i infile : The file containing bytes to be decoded (default is stdin)}
    \item{-o outfile : The output file to store decoded bytes (default is stdout)}
\end{itemize}

\subsection{Entropy}

The source code for this program is provided in the class resources repository.
Entropy calculates the entropy, or variation of contents, for a file.

\section{Parse}

Program arguments are parsed and stored using flags.
A set is not used since the order of the inputs is irrelevant and the parsed data
will ultimately be stored in variables anyway.

After program arguments are parsed the input file is parsed with encode or decode.
Execution happens with every grab of a byte to the buffer.
Refer to execute for how the algorithm works.

\section{Execute}

\subsection{Encode}

For encode, the parsed byte is divided into an upper and lower nibble. 
Then, the lower nibble is encoded and the byte is written to file out.
The same is done for the upper byte.
This process is repeated until the entire file is parsed. The pseudocode
is shown in \vref{encode}.

\begin{algorithm}
    \caption{Encode}\label{encode}
\end{algorithm}

Encoding a nibble requires 3 main steps: converting a nibble to a bit
vector, multiplying by the encoding matrix, then converting the resulting
matrix back into a byte. This process requires a BitMatrix implementation which then 
requires a BitVector implementation to function.
Both the Bitvector and BitMatrix functions are explained in detail in the lab document
so they will not be discussed here.

\subsection{Decode}

Decode is (unsurprisingly) the opposite of encode. After one byte is
parsed, it is decoded and stored in a variable last.
Every even odd iteration, a byte is parsed and decoded like before,
but instead of storing in last, the function pack\_byte is called
and puts the last nibble and current nibble together and writes to
stdout.
The pseduocode is shown in \vref{decode}.

\begin{algorithm}
    \caption{Decode}\label{decode}
\end{algorithm}

\end{document}
