\documentclass[12pt]{article}
\usepackage{charter} % font
\usepackage[margin=1in]{geometry} % margin
\usepackage{varioref} % "on the next page" label decoration
\usepackage{hyperref} % hyperlinks
\usepackage{enumitem} % Enumeration
\usepackage{float} % table placement

\usepackage[lined,dotocloa]{algorithm2e} % Pseudocode
\labelformat{algocf}{\textit{alg.}\,(#1)}

% Questions
\newenvironment{QandA}{\begin{enumerate}[label=\bfseries\alph*.]\bfseries}
{\end{enumerate}}
\newenvironment{answered}{\par\normalfont}{}
\usepackage{lipsum}

%%%%%%%%%%%%%%%%%%%%%%%%%%%%%%%%%%%%%%%%%%%%%%%%%%%%%%%%%%%%%%%%%%%%%%%%%%%%%%%%

\title{%
	\textbf{Assignment 5 \\ 
	Hamming Codes \\
	\large DESIGN DOCUMENT} }
	\author{Zack Traczyk \\ CSE13S - Spring 2021}
	\date{Due: May 9\textsuperscript{th} at 11:59 pm}

%%%%%%%%%%%%%%%%%%%%%%%%%%%%%%%%%%%%%%%%%%%%%%%%%%%%%%%%%%%%%%%%%%%%%%%%%%%%%%%%

\begin{document}

\maketitle

\section{Objective}


\section{Given}

\begin{itemize}
	\item{Header files for stack, path, and graph}
	\item{Pseudocode for recursive search}
\end{itemize}

%%%%%%%%%%%%%%%%%%%%%%%%%%%%%%%%%%%%%%%%%%%%%%%%%%%%%%%%%%%%%%%%%%%%%%%%%%%%%%%%

\section{Prelab Questions}

\begin{QandA}

\item Completed Lookup table

	\begin{answered}

		Table \vref{lookup} shows which error vector values are associated with
		which bits/ error codes.

		\begin{table}[]
			\centering
			\begin{tabular}{l|l}
				\textbf{Value} & \textbf{Bit} \\ \hline
				0              & 0            \\
				1              & 4            \\
				2              & 3            \\
				3              & HAM\_ERR     \\
				4              & 2            \\
				5              & HAM\_ERR     \\
				6              & HAM\_ERR     \\
				7              & 5            \\
				8              & 1            \\
				9              & HAM\_ERR     \\
				10             & HAM\_ERR     \\
				11             & 6            \\
				12             & HAM\_ERR     \\
				13             & 7            \\
				14             & 8            \\
				15             & HAM\_ERR    
			\end{tabular}
			\caption{Lookup Table}\label{lookup}
		\end{table}

	\end{answered}

	\item Decode the following codes. If it contains an error, show and explain
		how to correct it.

	\begin{answered}
		1110 0011 -> $e$ = (1233) = (1011) = 1101 = 11 = Bit 6
		1101 1000 -> $e$ = (2121) = (0101) = 1010 = 10 = HAM\_ERR
	\end{answered}

\end{QandA}

%%%%%%%%%%%%%%%%%%%%%%%%%%%%%%%%%%%%%%%%%%%%%%%%%%%%%%%%%%%%%%%%%%%%%%%%%%%%%%%%

\section{Test Harness}

\begin{itemize}
	\item{h : Command line options }
	\item{v : Verbose printing; prints all Hamiltonian paths found as well
		as total number of recursive calls to dfs()}
	\item{i : infile; the input file containing the cities and edges of a graph (default should be stdin)}
	\item{o : outfile; the output file to print to (default is stdout)}
\end{itemize}

\subsection{Parse}

\subsection{Execute}

\section{Algorithm implementation}

\begin{algorithm}
	increment calls counter\;
	\If{first call}{
		add vertex to path according to graph\;
		}
		mark vertex as visited in graph\;
		edge number = adjacent\_edges(graph, vertex to check, array to store in)\;
		\eIf{path has hit every node}{
			push vertex to path\;
			\If{path is shortest}{
				\If{verbose argument set}{
					print path\;
					}
					copy shortest path\;
					}
					pop vertex from path\;
					}{
						\For{every adjacent vertex}{
							\If{edge not visited}{
								push adjacent vertex to path\;
								\If{if path length is not longer than shortest}{
									recursive call to DFS\;
									}
									pop adjacent vertex from path\;
									}
									}
									}
									add vertex to path according to graph\;
									\caption{DFS}\label{execute}
\end{algorithm}

\end{document}
