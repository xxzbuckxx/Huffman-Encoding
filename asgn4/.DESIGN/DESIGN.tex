\documentclass[12pt]{article}
\usepackage{charter} % font
\usepackage[margin=1in]{geometry} % margin
\usepackage{varioref} % "on the next page" label decoration
\usepackage{hyperref} % hyperlinks
\usepackage{enumitem} % Enumeration

\usepackage[lined,dotocloa]{algorithm2e} % Pseudocode
\labelformat{algocf}{\textit{alg.}\,(#1)}

%%%%%%%%%%%%%%%%%%%%%%%%%%%%%%%%%%%%%%%%%%%%%%%%%%%%%%%%%%%%%%%%%%%%%%%%%%%%%%%%

\title{%
\textbf{Assignment 4 \\ 
The Circumnavigations of Denver Long \\
\large DESIGN DOCUMENT} }
\author{Zack Traczyk \\ CSE13S - Spring 2021}
\date{Due: May 2\textsuperscript{nd} at 11:59 pm}

%%%%%%%%%%%%%%%%%%%%%%%%%%%%%%%%%%%%%%%%%%%%%%%%%%%%%%%%%%%%%%%%%%%%%%%%%%%%%%%%

    \begin{document}

    \maketitle

	\section{Objective}

	The program (tsp) calculates the shortest route between a given vertices.
	The program can optionally read and write to a file.

	\section{Given}

	\begin{itemize}
		\item{Header files for stack, path, and graph}
		\item{Pseudocode for recursive search}
	\end{itemize}

%%%%%%%%%%%%%%%%%%%%%%%%%%%%%%%%%%%%%%%%%%%%%%%%%%%%%%%%%%%%%%%%%%%%%%%%%%%%%%%%

	\section{Test Harness}

	Command arguments:

	\begin{itemize}
		\item{h : Command line options }
		\item{v : Verbose printing; prints all Hamiltonian paths found as well
			as total number of recursive calls to dfs()}
		\item{u : Specifes the graph to be undirected}
		\item{i : infile; the input file contaiing the cities and edges of a graph (default should be stdin)}
		\item{o : outfile; the output file to print to (default is stdout)}
	\end{itemize}

	\subsection{Parse}

	No need to use a set, just set defaults than change on parse.

	After parsing program arguments, input file is parse.
	The first line grabs number of iterations to store string pointers of strings on following lines.

	Next a graph is constructed and the following vertices are stored.

	\subsection{Execute}

	Calls ADT in search.c passing the parsed array of string pointers and graph.

	\section{Algorithm implementation}

	\subsection{Stack, Path, Graph}

	Given the pseudocode stack, path, and graph implementation is straight forward.

	\subsection{Search}

	search.c contains two functions: find adjacent edges given a vertex, and recursively find a Hamiltonian path (DFS).

	Adjacent edges are found by finding seeing what edges exist in given row (vertex) in graph.

	ADT uses the psedocode in the lab doc.

	\end{document}
